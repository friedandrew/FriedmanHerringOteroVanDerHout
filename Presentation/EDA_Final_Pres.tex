% Options for packages loaded elsewhere
\PassOptionsToPackage{unicode}{hyperref}
\PassOptionsToPackage{hyphens}{url}
%
\documentclass[
  ignorenonframetext,
]{beamer}
\title{Washington Dams and AIM Gage Sites}
\author{Jackie Van Der Hout, Andrew Friedman-Herring, \& Catherine
Otero}
\date{4/11/2022}

\usepackage{pgfpages}
\setbeamertemplate{caption}[numbered]
\setbeamertemplate{caption label separator}{: }
\setbeamercolor{caption name}{fg=normal text.fg}
\beamertemplatenavigationsymbolsempty
% Prevent slide breaks in the middle of a paragraph
\widowpenalties 1 10000
\raggedbottom
\setbeamertemplate{part page}{
  \centering
  \begin{beamercolorbox}[sep=16pt,center]{part title}
    \usebeamerfont{part title}\insertpart\par
  \end{beamercolorbox}
}
\setbeamertemplate{section page}{
  \centering
  \begin{beamercolorbox}[sep=12pt,center]{part title}
    \usebeamerfont{section title}\insertsection\par
  \end{beamercolorbox}
}
\setbeamertemplate{subsection page}{
  \centering
  \begin{beamercolorbox}[sep=8pt,center]{part title}
    \usebeamerfont{subsection title}\insertsubsection\par
  \end{beamercolorbox}
}
\AtBeginPart{
  \frame{\partpage}
}
\AtBeginSection{
  \ifbibliography
  \else
    \frame{\sectionpage}
  \fi
}
\AtBeginSubsection{
  \frame{\subsectionpage}
}
\usepackage{amsmath,amssymb}
\usepackage{lmodern}
\usepackage{iftex}
\ifPDFTeX
  \usepackage[T1]{fontenc}
  \usepackage[utf8]{inputenc}
  \usepackage{textcomp} % provide euro and other symbols
\else % if luatex or xetex
  \usepackage{unicode-math}
  \defaultfontfeatures{Scale=MatchLowercase}
  \defaultfontfeatures[\rmfamily]{Ligatures=TeX,Scale=1}
\fi
% Use upquote if available, for straight quotes in verbatim environments
\IfFileExists{upquote.sty}{\usepackage{upquote}}{}
\IfFileExists{microtype.sty}{% use microtype if available
  \usepackage[]{microtype}
  \UseMicrotypeSet[protrusion]{basicmath} % disable protrusion for tt fonts
}{}
\makeatletter
\@ifundefined{KOMAClassName}{% if non-KOMA class
  \IfFileExists{parskip.sty}{%
    \usepackage{parskip}
  }{% else
    \setlength{\parindent}{0pt}
    \setlength{\parskip}{6pt plus 2pt minus 1pt}}
}{% if KOMA class
  \KOMAoptions{parskip=half}}
\makeatother
\usepackage{xcolor}
\IfFileExists{xurl.sty}{\usepackage{xurl}}{} % add URL line breaks if available
\IfFileExists{bookmark.sty}{\usepackage{bookmark}}{\usepackage{hyperref}}
\hypersetup{
  pdftitle={Washington Dams and AIM Gage Sites},
  pdfauthor={Jackie Van Der Hout, Andrew Friedman-Herring, \& Catherine Otero},
  hidelinks,
  pdfcreator={LaTeX via pandoc}}
\urlstyle{same} % disable monospaced font for URLs
\newif\ifbibliography
\setlength{\emergencystretch}{3em} % prevent overfull lines
\providecommand{\tightlist}{%
  \setlength{\itemsep}{0pt}\setlength{\parskip}{0pt}}
\setcounter{secnumdepth}{-\maxdimen} % remove section numbering
\ifLuaTeX
  \usepackage{selnolig}  % disable illegal ligatures
\fi

\begin{document}
\frame{\titlepage}

\begin{frame}
More info on slides in R here:
\url{https://bookdown.org/yihui/rmarkdown/ioslides-presentation.html}
\end{frame}

\begin{frame}{\textbf{Research Questions}}
\protect\hypertarget{research-questions}{}
\begin{itemize}
\tightlist
\item
  How many dams are upstream of the gage sites?
\item
  How do the variables measured at each gage site correlate?
\item
  How do the number of dams correlate with the AIM variables?
\end{itemize}

\begin{block}{\emph{Why ask?}}
\protect\hypertarget{why-ask}{}
\begin{itemize}
\tightlist
\item
  See the extent of the damming in Washington
\item
  See the impact of dams on AIM variables
\end{itemize}
\end{block}
\end{frame}

\begin{frame}{\textbf{What data did we use?}}
\protect\hypertarget{what-data-did-we-use}{}
\begin{itemize}
\tightlist
\item
  BLM Assessment, Inventory, and Monitoring (AIM) Data
\item
  Washington State Department of Ecology Inventory of Statewide Dams
\item
  National Hydrography Dataset of Flowlines and Catchments, developed by
  the US EPA and USGS
\end{itemize}
\end{frame}

\begin{frame}{\textbf{Data Wrangling}}
\protect\hypertarget{data-wrangling}{}
\begin{itemize}
\tightlist
\item
  Data scraping publicly available geospatial data for AIM and dam data
\item
  Using NHDplustools package to extract hydro data from the NHD
\end{itemize}
\end{frame}

\begin{frame}{\textbf{Analysis of the Data}}
\protect\hypertarget{analysis-of-the-data}{}
\begin{itemize}
\tightlist
\item
  Analytical approaches
\item
  Correlations among and within variables
\item
  Geospatial tools to assess number of dams in catchments flowing into
  sample sites
\end{itemize}
\end{frame}

\begin{frame}{\textbf{Results of the Analysis}}
\protect\hypertarget{results-of-the-analysis}{}
\end{frame}

\begin{frame}{Where are the AIM sample sites and their catchments?}
\protect\hypertarget{where-are-the-aim-sample-sites-and-their-catchments}{}
\end{frame}

\begin{frame}{Where are all of dams in Washington State?}
\protect\hypertarget{where-are-all-of-dams-in-washington-state}{}
\end{frame}

\begin{frame}{Catchments, Dams, and Sample Sites}
\protect\hypertarget{catchments-dams-and-sample-sites}{}
\end{frame}

\begin{frame}{Water Quality Correlation Plot}
\protect\hypertarget{water-quality-correlation-plot}{}
\end{frame}

\begin{frame}{Biodiversity and Riparian Habitat Quality Correlation
Plot}
\protect\hypertarget{biodiversity-and-riparian-habitat-quality-correlation-plot}{}
\end{frame}

\begin{frame}{Watershed Function and Instream Habitat Quality
Correlation Plot}
\protect\hypertarget{watershed-function-and-instream-habitat-quality-correlation-plot}{}
\end{frame}

\begin{frame}{\textbf{Next Steps}}
\protect\hypertarget{next-steps}{}
\begin{itemize}
\tightlist
\item
  Running an AIC
\item
  Running regression
\item
  Calculating flowline distances with NHDtoolsplus
\item
  Comparing on AIM variables between sample site catchments on the basis
  of \# of dams, dams / square mile, and flowline distance to dams
\end{itemize}
\end{frame}

\end{document}
